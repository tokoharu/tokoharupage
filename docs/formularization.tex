\documentclass[13pt, a4paper, landscape]{jarticle}


\title{LPとグラフと定式化}
\author{tokoharu}

\date {\today}

\usepackage{amsthm}
\usepackage{algorithm}
\usepackage{algorithmic}
\usepackage{ascmac}
\usepackage{amsmath}
\usepackage{courier}
%\usepackage{eqnarray}
\usepackage[dvipdfmx]{graphicx}
\usepackage{helvet}
\usepackage{times}
\setlength{\textwidth}{17cm}
\setlength{\textheight}{23cm}
\setlength{\oddsidemargin}{-0.5cm}
\setlength{\evensidemargin}{-0.5cm}
\setlength{\topmargin}{-1cm}
%\usepackage{ascmac}
%\usepackage{here}
\usepackage{txfonts}
\usepackage{listings}
\usepackage{lscape}
\usepackage{url}
%\usepackage{hyperref}

\renewcommand{\lstlistingname}{Liblary}

\renewcommand{\proofname}{\bfseries proof}

\newtheoremstyle{nonitalic}% name
  {}%      Space above, empty = `usual value'
  {}%      Space below
  {\normalfont}% Body font
  {}%         Indent amount (empty = no indent, \parindent = para indent)
  {\bfseries}% Thm head font
  {. \\ }%        Punctuation after thm head
  { }%     Space after thm head: " " = normal interword space
       	%         \newline = linebreak
  {}% Thm head spec
\theoremstyle{nonitalic} % from http://d.hatena.ne.jp/su48us/comment?date=20131206
\newtheorem{Definition}{定義}[section]
\newtheorem{Theorem}{定理}[section]
\newtheorem{Problem}{問題}[section]
%\theoremclass{Definition}
%\theoremstyle{break}
%\newframedtheorem{Def}[Definition]{定義}

\begin{document}
\maketitle

\section{LPと主要問題}
LP(Linear Programming)とは線形計画問題のことである.
\subsection {LP, LPの双対問題}
まずはLPの標準的な形を記述する.$A$は定数行列,$b, c$は定数ベクトル,ベクトルに対する不等式は全ての要素に対してその不等式が成立することを表す.

LP標準形
\begin{align}
 &&&&&\textrm{maximize}   && c^Tx \\
 &&&&&\textrm{subject to} && Ax \leq b  &&&&&\\
 &&&&&                    && x_i \geq 0 & i = 1,...,n &&&&&
\end{align}

双対LP標準形
\begin{align}
 &&&&&\textrm{minimize}   && b^Ty \\
 &&&&&\textrm{subject to} && A^T y \geq c  &&&&&\\
 &&&&&                    && y_j \geq 0 & j = 1,...,m &&&&&
\end{align}

 
$A$は$m\times n$行列, 
$c,x$は$n$次元列ベクトル, 
$b,y$は$m$次元列ベクトルになっている. \\

例 : 
\begin{equation*}
 A = 
 \left[\begin{array}{cc}
  1 & 1 \\
  5 & 2 \\
 \end{array}\right],
 b = 
 \left[\begin{array}{c}
     1 \\
     3 \\
   \end{array}\right],
 c = 
 \left[\begin{array}{c}
     2 \\
     1 \\
 \end{array}\right]
\end{equation*}
のとき最適解 $x = (1/3, 2/3) , \gamma = 4/3  $, を得る.図を描けばわかりやすい.これの双対LP解は
$y = (1/3, 1/3), \gamma = 4/3$である($\gamma$は最適値を表す).
このように主問題に最適解が存在すれば双対問題にも最適解が存在し,主問題の最適値と双対問題の最適値は一致する(強双対定理).

\subsubsection{LPと双対LPの変換のための表}

最小化問題と最大化問題をそれぞれその双対に変換するときに変換する時には下の表を使うとやりやすい.

\begin{equation*}
\begin{array}{c|c}
  \textrm{最小化問題} &  \textrm{最大化問題} \\ \hline \hline
  \textrm{変数} &  \textrm{制約式} \\ \hline
  \leq & \geq \\
  \textrm{自由} & = \\
  \geq & \leq \\ \hline \hline
  \textrm{制約式} & \textrm{変数} \\ \hline
  \leq & \leq \\
   =   & \textrm{自由} \\
  \geq & \geq
\end{array}
\end{equation*}


\subsection{最短経路問題とLP}

この節では次の3つのトピックを紹介する \\
1. 自己ループ無し最短経路問題について \\
2. LPで解いた時の解とBellman-Fordで解いた時の解の差\\
3. 2を実際の問題で気を付ける必要があるか
\\

 1について.自己ループが無ければ最短経路問題はLPの形で書くことができる.発想としては後述する最小費用流の特殊ケースと思えば良い.さらにこのLPの双対を考えることができ、かなりシンプルな形の式が出てくる.競技プログラミング界隈では差分制約,俗に牛ゲーと呼ばれる.これはPOJ3169(Layout)に登場する牛から来ている. \\

また,これ以後$\delta_{v}^+$は頂点$v$から出る辺の集合を表し,$\delta_{v}^-$は頂点$v$に入ってくる辺の集合を表すものとする.

LP主問題
($|E|$変数, $|V|$制約)
\begin{align}
 &&&&&\textrm{minimize}   && \sum_{e  \in E} c_e y_e \\
 &&&&&\textrm{subject to} && \sum_{e \in \delta_v^- } y_e - \sum_{e \in \delta_v^+} y_e \geq 0 & v\neq s, v\neq t  &&&&&\\
 &&&&&                    && \sum_{e \in \delta_v^- } y_e - \sum_{e \in \delta_v^+} y_e \geq 1 & v = t  &&&&&\\
 &&&&&                    && \sum_{e \in \delta_v^- } y_e - \sum_{e \in \delta_v^+} y_e \geq -1 & v = s  &&&&&\\
 &&&&&                    &&  y_e \geq 0 & e \in E&&&&&
\end{align}

LP双対問題
($|V|$変数, $|E|$制約)
\begin{align}
 &&&&&\textrm{maximize}   && x_t - x_s \\
 &&&&&\textrm{subject to} && x_j - x_i\leq c_e & e = (i,j) \in E  &&&&&\\
 &&&&&                    && x_v \geq 0 & v \in V &&&&&
\end{align}
2.について.\\
 Bellman-Fordを走らせた場合とLPで解いた場合の差異を考える.この時に重要なのはBellman-Fordの実装であるが,初期値として$s$のみ距離0としてそれ以外の全ての頂点の距離をINFとしておき,更新が起きれば順次更新する、というようにする.このBellman-Fordを走らせた時に負閉路が検出されれば双対問題は解が存在せず,主問題では負へ発散となる.同じくBellman-Fordを走らせた時に$x_t$がINFのままなら,双対問題では正へ発散,主問題では解なしとなる.

しかし,同値ではない可能性に注意が必要となる.気を付けなければいけないのは$s$から$t$の経路に含まれ得ないような負閉路が存在するときである.この時にはLPは主問題で負へ発散,双対問題では解なしになる一方,Bellman-Fordでは$x_t$の値が実装方法によっては正しく出てしまう.

したがって,解きたいLPが差分制約の形をしていた時,それを最短経路用のグラフに直して最短経路問題を解くことになるが,この時には正確な$s-t$最短経路を求めずに負閉路発見を最優先にしなければならない.(著者もこのあたりは怪しいので間違っている可能性があります.間違っている場合には教えてください)\\

3.について.

POJ3169を見てみる.これをグラフに直すと頂点$N$から任意の頂点は到達可能であることがわかる.したがって,負閉路がどこかに存在すれば[$1, \dots ,N,\dots, ( \textrm{負閉路を何周もする} ),\dots ,1, \dots, N$]という負閉路を含む経路を構成できる.したがって2で危惧されたような$s-t$経路に含まれ得ないような負閉路ということは存在しないことになる.つまり,2のようなことは考えずに最短経路を解いてしまえば良い.

パソコン甲子園2014予選問10を見てみる.これは先程の場合とは異なり,$t$の指定がない.先頭が1であることから,始点は頂点$1$で,頂点$1$は任意の頂点から到達可能であることがわかる.この場合には,ある頂点$v$を終点と固定した時に$s-v$経路以外にループが出来うるかというと,でき得る.具体的には頂点1から到達不可能なところに負閉路を置くとできる.したがって$v$を終点と固定した時には$s-v$最短経路と最適解が異なってしまう.この問題の場合にはそこに気をつけて最初に負閉路判定をし,それがなければ頂点1から到達不可能な頂点が存在するかどうかを判定する,というようにしなければならないと思われる.(著者は実際にこの問題を解いたわけではないので間違っているかもしれない)



\subsection{最大流問題とLP}
最大流最小カット定理は有名な定理だが,ここではその定式化と実際に最大流最小カット定理がどのように双対になっているかを見る.

与えられる定数は$c_{ij}$であり,これは辺$ij$に流せる最大流量を意味する.

LP主問題 \\
新たに辺$ts$を追加して$x_{ts}$を最大化する最大循環流問題として考える. \\
したがって辺$ts$を辺集合$E$に付け加えて$E'$としておく.\\
この状況下で $|E|+1$変数 $|E|+|V|$制約の次の式で書ける.
\begin{align}
 &&&&&\textrm{maximize}   && x_{ts}  \\
 &&&&&\textrm{subject to} && x_e\leq c_e & e \in E  &&&&&\\
 &&&&&                    && \sum_{e \in \delta_v^- \subset E'} x_e - \sum_{e \in \delta_v^+  \subset E'} x_e \leq 0 & v \in V &&&&& \\
 &&&&&                    && x_e \geq 0 & e \in E' &&&&&
\end{align}
$E, E'$がそれぞれ登場していることに注意.


LP双対問題 \\
主問題における$v$に関する不等式に対応する双対問題の変数が$p$であり,主問題における$e$に関する不等式に対応する双対問題の変数が$y$であるとする.\\
この時次のような$|E|+|V|$変数 $|E|+1$制約の次の式が書ける.
\begin{align}
 &&&&&\textrm{minimize}   && \sum_{e \in E} c_e y_e  \\
 &&&&&\textrm{subject to} && y_e + p_j - p_i \geq 0 &  e = (i, j) \in E  &&&&&\\
 &&&&&                    && p_s - p_t  \geq 1         &&&&& \\
 &&&&&                    && p_v \geq 0,  y_e \geq 0 &  v \in V, e \in E &&&&&
\end{align}

これが最小カットになることはなんとなくはわかるが真面目な証明をしようとすると厳密性に自信がない.ひとまず説明をしようすると次のようになる.

$y_e$をあまり大きくしたくないことを考えると$p$で生じる差はできるだけ小さくしたくなり,$p_s-p_t=1$の場合が最適ということはわかる.さらに$p_s, p_t$以外の数が$p$に登場しても嬉しくない.なので平行移動させて$p_s=1, p_t=0$としてよい.これで頂点に対応する$p_i$のうちどれを1にしてどれを0にするかという問題になり $y_e$ はギリギリのところまで抑えるのがよいのでこれも0,1になり,結局最小カット問題になる.


\subsection{最小費用流とLP}

最小費用流も有名であるのでLPとして書いてみる. \\
この問題では辺に対して3種類の定数が与えられる. 最初の2つは$c_{ij}, u_{ij}$であり,それぞれ $ij$間に1流すコスト, $ij$間に流せる流量を表す. 
もう1種類は$st$間に流せる量$F$である.

最大流問題のときと同じように $E' = E \cup \{ (t,s) \} $としておく.

\begin{align}
 &&&&&\textrm{minimize}   && \sum_{e\in E} c_e x_e  \\
 &&&&&\textrm{subject to} && x_{e} \leq u_e & e  \in E \\
 &&&&&                    && \sum_{e \in \delta_v^-  \subset E'} x_e - \sum_{e \in \delta_v^+  \subset E'} x_e= 0 & v \in V  &&&&& \\
 &&&&&                    && x_{ts} = F \\
 &&&&&                    && x_e \geq 0 & e \in E &&&&&
\end{align}



\section{ネットワークフローでの定式化}
\subsection{Maximum Closure Problem / Project Selection Problem}
Maximum Closure Problem~(もしくはProject Selection Problem) とは次のような問題である.\\

$N$個の要素がある.最初どの頂点も集合$B$に属しているが,これを集合$A$に移すことで利益を最大化したい. \\
要素$i$が$A$に属する時には利得 $p_i$ を得る情報が与えられる. \\
さらに3つ組$(x_j, y_j, z_j)$が与えられ,これは$x_j$が$A$に属し,かつ$y_j$が$B$に属していた時に$z_j$だけ損失をすることを意味する. \\
この時に得られる最大の利得を答えよ. \\


この問題に対する解は
\[ \sum_{v \in V } \max(0,p_v) - \mathrm{maxflow}(s,t) \]
で求めることができる.\\
ただし $\mathrm{maxflow(s,t)} $とは\\
1.要素に対応する点の他に新しく$s,t$の頂点を導入した$N+2$頂点のグラフの上に\\
2.$p_v>0$であれば$s$から$v$に容量$p_v$の辺を張り\\
3.$p_v<0$であれば$v$から$t$に容量$-p_v$の辺を張り \\
4.$x_j$から$y_j$に容量$z_j$の辺を張ったときの\\
5.$st$間の最大流の値である.


最小カットを用いて解く問題で難しい問題はこの問題を通して考えると見通し良く考えることができる場合があるので有用である.



\subsection{有理数フローについて注意}

下記Problemsでも紹介するとおり, 有理数の最大流問題の解を求めなければならない場合がある. 単純にDinicのアルゴリズムにおいてintをdoubleに変更すれば良いかというと実際には調整が難しい面がある(過去に泥沼にハマった経験がある). したがって整数フローに直して解く方が正確性も勝るので良い.

特に二分探索をするのであれば,分母と分子でそれぞれ整数の変数を持つと実装がしやすい.



\section{Problems}

\subsection{AOJ2230; How to Create a Good Game}

問題概要 : DAGが与えられる.DAGの最大長を維持しながら辺に長さを足す.足せる長さの合計の最大値を答えよ.\\

とりあえずLPとして記述してみる.このとき次のような形になる.$|E|+|V|$変数, $|E|$ + 1 制約.
\begin{align}
 &&&&&\textrm{maximize}   && \sum_{e \in E} a_e \\
 &&&&&\textrm{subject to} && a_e - x_j + x_i \leq - c_e & e = (i,j) \in E  &&&&&\\
 &&&&&                    && x_t \leq  D &&&&&& \\
 &&&&&                    && x_v \geq 0, a_e \geq 0 & v \in V, e \in E &&&&&
\end{align}
このLPに対して双対を取ってみる.

\begin{align}
  &&&&& \textrm{minimize}   && Dp_t - \sum_{e \in E} c_e y_e \\
  &&&&& \textrm{subject to} && -\sum_{e \in \delta_v^- } y_e + \sum_{e \in \delta_v^+ } y_e \geq 0 & v \in V, v\neq t &&&&&\\
  &&&&&                     && p_t - \sum_{e \in E} y_e \geq 0 & v=t \\
  &&&&&                     && y_e \geq 1  & e \in E\\
  &&&&&                     && y_e \geq 0, p_t \geq 0 & e\in E
\end{align}


これは$p_t$を元々のグラフの頂点$t$から$t'$への辺の容量と解釈することで,最低流量1の$s-t'$最小費用流であると解釈できる.

\subsection{ジョブ割当問題}
問題概要 : $n$個のジョブと$m$人の作業員がいる.
$S_i$はジョブ$i$を実行できる作業員の集合である.
以下の様な定式化のもとで$x_{ij}$としてありうる最適解をひとつ求めよ.
\begin{align}
  &&&&& \textrm{minimize}   && \max_{j=1,...,m} \sum_{i:j \in S_i} x_{ij} \\
  &&&&& \textrm{subject to} && \sum_{j \in S_i} x_{ij} = t_i  & i=1,...,n &&&&&\\
\end{align}\\


まず,maxが出現した時の常套手段はそれに変数をつけてmaxの構成要素を制約式に持ってくることなので次のように式変形をする.
\begin{align}
  &&&&& \textrm{minimize}   && T \\
  &&&&& \textrm{subject to} && \sum_{j \in S_i} x_{ij} = t_i  & i=1,...,n &&&&&\\
  &&&&&                     &&  \sum_{i:j \in S_i} x_{ij} \leq T & j=1,...,m
\end{align}

この式で,ある値$T$を決め打ちした時に解が存在するかどうかは判定が可能である. なぜなら,頂点$s$と頂点$i$の間に容量$t_i$の辺を張り,$i$と$j (\in S_i)$の間に容量無限大の辺を張り,$j$と$t$の間に容量$T$の辺を張って,$\sum_{i=1}^n t_i $ だけ最大流が流れるかどうかをチェックすればよいからである.

あとは$T$を二分探索すれば最適解を見つけることができる.


\subsection{POJ3155; Hard Life}
問題概要 : 無向グラフ$G$が与えられる.この部分グラフの中で(辺数) / (頂点数)を最大化されるような部分グラフを求めよ(頂点数$\leq$100, 辺数$\leq$1000)\\

二分探索で解を求めることを考える.すなわち,$ (\textrm{辺数}) / (\textrm{頂点数})  > x$なるような部分グラフが存在するかどうかを判定する.

式変形をすると
\begin{equation}
  (\textrm{辺数}) - x (\textrm{頂点数}) > 0 \label{ineqHardLife}
\end{equation}
なる部分グラフを持つかを判定する問題になる.

(\ref{ineqHardLife})式の最大値が0より大きいことを判定することはProject Selection Problemを適用すれば判別できる. \footnote{ Project Selection Problemにおいて最適値は0以上の値になる.(\ref{ineqHardLife})式の最大値は,部分グラフとして空のグラフを選べると考えれば0以上になる.しかし空グラフは元の問題の定義では定義ができない.したがって,最適値が0になるような$x$では元の問題の実行可能でない解である可能性がある.そのため,$ (\textrm{辺数}) - x (\textrm{頂点数}) \leq 0 $  という式で考えようとしたり,他の問題でぴったり$0$になるときのグラフを復元しようと思った時には苦労する可能性がある.}

具体的には辺の集合と頂点の集合を合わせた集合から利得最大になるように要素を選ぶ問題という風に考えて,辺を選ぶと利得が+1される,頂点を選ぶと利得が $-x$ されるという風に解釈すれば良い.さらに,ある辺を使用すればその両端点は使用しなければならないがこの制約もProject Selection Problemで表現可能である. 

したがって(パラメータ$x$によって作られたProject Selection Problemの解) $>0$を判定する二分探索を実装すれば良い. 


%\subsection{AOJ2571; Floating Islands}



\section*{謝辞}
この文章を書く際,Mi\_Sawaさんとuwiさんに文章の添削や助言をして頂きました.この場を借りて感謝いたします.

\section*{編集履歴}
\begin{align*}
  &&& 2014/10/17 && アジア地区予選用のライブラリの一部として書く &&&\\
  &&&            && \mathrm{LP}, 最短経路, 最大流, 最小費用流, \textrm{Maximum Closure Problem}, \\
  &&&            &&\textrm{How to create a good game}, ジョブ割当問題を書く \\
  &&& 2014/12/14(?) && 有理数フロー, \textrm{Hard Life}を追記する。 \\
  &&& 2014/12/22    && 記事の修正(主に細かいところ)\\
  &&& 2014/12/31    && 最短経路の記事を大幅修正.\textrm{LP}と最短経路の橋渡しで怪しいところを明確に記述.\\
  &&&               && 辺の集合を表す記法として \delta を用いることにした\\
\end{align*}


\end{document}


